
\section{Query Cardinality}
\label{sec:cardinality}

\begin{figure*}
    \label{fig:hierarchical-form}
    \centering
    \begin{forest}
        void/.style = {
            draw, circle},
        map/.style = {
            draw, rectangle, rounded corners=2mm,
            minimum width=1.5cm, minimum height=.5cm,
            text height=1.5ex, text depth=.25ex,
            font=\small\sffamily},
        blank/.style = {
            no edge},
        more/.style = {
            rectangle, minimum height=.5cm,
            edge={-{stealth'}}},
        singular/.style = {
            edge={-}},
        optional/.style = {
            edge={-{Circle[open,length=3.5pt]}}},
        plural/.style = {
            edge={-{Straight Barb[reversed,length=3.5pt]}, shorten >=-0.3pt}},
        for tree = {
            parent anchor=south,
            child anchor=north,
            edge path={
                \noexpand\path[\forestoption{edge}](!u.parent anchor) -- +(0,-4pt) -| (.child anchor)\forestoption{edge label};}}
        [{},void
            [department,map,plural
                [name,map,singular]
                [employee,map,plural
                    [name,map,singular]
                    [\dots,blank]
                    [department,map,singular
                        [\dots,more]
                        [\dots,more]]
                    [\dots,blank]
                    [subordinate,map,plural
                        [\dots,more]
                        [\dots,blank]
                        [\dots,more]]]]
            [employee,map,plural
                [name,map,singular]
                [position,map,singular]
                [salary,map,singular]
                [department,map,singular
                    [name,map,singular]
                    [employee,map,plural
                        [\dots,more]
                        [\dots,blank]
                        [\dots,more]]]
                [manager,map,optional
                    [\dots,more]
                    [\dots,blank]
                    [\dots,more]]
                [subordinate,map,plural
                    [\dots,more]
                    [\dots,blank]
                    [\dots,more]]]]
    \end{forest}
    \caption{Database schema in hierarchical form}
\end{figure*}

In Section~\ref{sec:introduction}, we suggested that any function defined on an
entity set can be seen as a query.  However, this query model failed to
represent optional and plural relationships as well as queries without apparent
input.  In this section, we resolve these issues by introducing the notion of
\emph{query cardinality}.

We found it difficult to model these two relationships:

\begin{enumerate}[(i)]
\item \label{itm:employee-to-manager}
\emph{An employee may report to another employee, the manager.}

\item \label{itm:department-to-employee}
\emph{Any department has a number of associated employees.}
\end{enumerate}

We were also puzzled on how to express input-free queries such as:

\begin{enumerate}[(i)]
\setcounter{enumi}{2}
\item \label{itm:department-set}
\emph{Show a list of all departments.}
\end{enumerate}

Let us start with relationships~(\ref{itm:employee-to-manager})
and~(\ref{itm:department-to-employee}).  Expressing them as queries with
signatures
\begin{equation*}
    \Emp \to \Emp, \qquad \Dept \to \Emp
\end{equation*}
would imply exactly one output value for any input.  But an employee may have
no manager, and a department may have more than one employee.

In programming practice, optional and plural values are stored using container
types.  So let us define two parametric types: $\Opt{A}$ as a zero or one value
of type $A$ and $\Seq{A}$ as a type of finite $A$-valued sequences, that is
\begin{equation*}
\Opt{A} = \{\bot\} \sqcup A, \qquad
\Seq{A} = \{[\;]\} \sqcup A \sqcup A^2 \sqcup \ldots.
\end{equation*}

With these, we can express relationships~(\ref{itm:employee-to-manager})
and~(\ref{itm:department-to-employee}) as primitives
\begin{alignat*}{3}
    & \Manager && : \Emp && \to \Opt{\Emp}, \\
    & \Employee && : \Dept && \to \Seq{\Emp}.
\end{alignat*}
The container structure over the query output is what we call the query
cardinality.

We can now guess the output type and cardinality of
query~(\ref{itm:department-set}).  Indeed, \emph{a list of all departments}
can only mean $\Seq{\Dept}$.

To describe the input this query, we need to introduce a \emph{singleton} type
\begin{equation*}
    \Void = \{\top\}.
\end{equation*}
Type $\Void$ has a unique inhabitant, and because there is no freedom in
choosing a value of this type, it can designate input that can never affect the
result of a query.  Thus, we can express~(\ref{itm:department-set}) with a
\emph{class} primitive
\begin{equation*}
    \Department : \Void \to \Seq{\Dept}
\end{equation*}
that produces all department entities.

Unfortunately, although containers let us represent optional and plural values,
they do not compose well.  For example, it is tempting to express a query that
\emph{for a given employee, finds their manager's salary} as a composition
\begin{equation} \label{eq:manager-to-salary}
    \Manager\To\Salary,
\end{equation}
or a query that \emph{shows the names of all departments} as
\begin{equation} \label{eq:department-to-name}
    \Department\To\Name.
\end{equation}
However, if we look at the signatures of the components
\begin{alignat*}{6}
    & \Manager && : \Emp && \to \Opt{\Emp}, \quad && \Salary && : \Emp && \to \Int, \\
    & \Department && : \Void && \to \Seq{\Dept}, \quad && \Name && : \Dept && \to \Text,
\end{alignat*}
we see that their intermediate domains do not agree, which means their
compositions are ill-formed.

To compose functions that carry extra structure with their output, we use a
concept of \emph{monadic composition} (\cite{Moggi1991}).

A monad is a parametric type $M\{A\}$ equipped with a generic function
$\operatorname{unit}_{M\{A\}} : A \to M\{A\}$ and a so-called monadic
composition rule that satisfy certain coherence axioms.  Monadic
composition of two functions
\begin{equation*}
    p : A \to M\{B\}, \qquad q : B \to M\{C\},
\end{equation*}
gives a function $p\,\To\,q$ with a signature of the same shape
\begin{equation*}
    p\,\To\,q : A \to M\{C\}.
\end{equation*}

Both $\Opt{A}$ and $\Seq{A}$ are monads.  Let us confirm that they admit
monadic composition.

Monadic composition of
\begin{equation*}
    p : A \to \Opt{B}, \qquad q : B \to \Opt{C}.
\end{equation*}
is just a composition of partial functions
\begin{equation*}
    p\,\To\,q : a \longmapsto \begin{cases}
        \bot & (p(a)=\bot\text{ or }q(p(a))=\bot), \\
        q(p(a)) & (\text{otherwise}).
    \end{cases}
\end{equation*}

To compose two sequence-valued functions
\begin{equation*}
    p : A \to \Seq{B}, \qquad q : B \to \Seq{C},
\end{equation*}
we first evaluate $p$ on the input $a$ of type $A$
\begin{equation*}
    a \overset{p}{\longmapsto} [b_1, b_2, \ldots],
\end{equation*}
then apply $q$ to every element of $p(a)$
\begin{equation*}
    [b_1, b_2, \ldots]
    \overset{[q]}{\longmapsto}
    [[c^{1}_{1}, c^{1}_{2}, \ldots], [c^{2}_{1}, c^{2}_{2}, \ldots], \ldots],
\end{equation*}
and finally erase the nested brackets
\begin{equation*}
    [[c^{1}_{1}, c^{1}_{2}, \ldots], [c^{2}_{1}, c^{2}_{2}, \ldots], \ldots]
    \overset{\cancel{\,[\;]\,}}{\longmapsto}
    [c^{1}_{1}, c^{1}_{2}, \ldots, c^{2}_{1}, c^{2}_{2}, \ldots].
\end{equation*}
The result is the value $(p\,\To\,q)(a)$ of type $\Seq{C}$.

Monadic (and regular) composition rules give us a way to compose queries of the
same cardinality.  To compose queries with mixed cardinalities, we promote
their cardinalities to a smallest common monadic container using a chain of
natural embeddings
\begin{equation*}
    A \hookrightarrow \Opt{A} \hookrightarrow \Seq{A}.
\end{equation*}
These embeddings are defined by
\begin{alignat*}{5}
    & \; && \bot && : \Opt{A} && \longmapsto [\;] && : \Seq{A}, \\
    & a : A \longmapsto\ && a && : \Opt{A} && \longmapsto [a] && : \Seq{A}.
\end{alignat*}

We can now state the query composition rule as follows: find the largest
cardinality of the components; lift all components to this cardinality; use
monadic composition.  For example, this rule gives queries
(\ref{eq:manager-to-salary}) and (\ref{eq:department-to-name}) signatures
\begin{alignat*}{3}
    & \Manager\To\Salary && : \Emp && \to \Opt{\Text}, \\
    & \Department\To\Name && : \Void && \to \Seq{\Int}.
\end{alignat*}

We are ready to present the design of a combinator-based query language.

\begin{description}
\item[Query interface:]
A database query is a function of the form
\begin{equation*}
    p : A \to M\{B\}.
\end{equation*}
We call $A$ its input type, $B$ its output type, and $M$ its monadic
cardinality.  $M\{B\}$ could be one of $B$, $\Opt{B}$ or $\Seq{B}$ and
the respective queries are called singular, optional or plural.

\item[Primitives:]
The set of primitives include class primitives
\begin{alignat*}{3}
    & \Department && : \Void && \to \Seq{\Dept}, \\
    & \Employee && : \Void && \to \Seq{\Dept},
\end{alignat*}
attributes
\begin{alignat*}{6}
    & \Name && : \Dept && \to \Text, \qquad
    && \Name && : \Emp && \to \Text, \\
    & \Position && : \Emp && \to \Text, \qquad
    && \Salary && : \Emp && \to \Int,
\end{alignat*}
and relationships
\begin{alignat*}{3}
    & \Department && : \Emp && \to \Dept, \\
    & \Employee && : \Dept && \to \Seq{\Emp}, \\
    & \Manager && : \Emp && \to \Opt{\Emp}, \\
    & \Subordinate && : \Emp && \to \Seq{\Emp}.
\end{alignat*}
Note that we allow multiple primitives to share the same name as long as they
do not share the same input.

\item[Combinators:]
Binary composition combinator takes two queries
\begin{equation*}
    p : A \to M_1\{B\}, \qquad
    q : B \to M_2\{C\}
\end{equation*}
and produces a query of the form
\begin{equation*}
    p\,\To\,q : A \to M\{C\}.
\end{equation*}
Here, $M = M_1 \lor M_2$ is the smallest common cardinality of $M_1$ and $M_2$.

More combinators will be described in the next section.
\end{description}

To visualize the full set of primitives, we could alter
Figure~\ref{fig:sample-schema} to add the $\Void$ node and missing arcs.  But
instead, we prefer to unfold it into an infinite tree, where each node stands
for a primitive and any two adjacent nodes are composable primitives (see
Figure~\ref{fig:hierarchical-form}).  We can say that
Figure~\ref{fig:hierarchical-form} represents the database in hierarchical
database model, which suggests us another model of a database query: a
transformation of a hierarchical database.

