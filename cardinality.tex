
\section{Query Cardinality}
\label{sec:cardinality}

In Section~\ref{sec:introduction}, we suggested that a database query could be
modeled as a function.  However, this na\"{\i}ve model failed to represent
optional and plural relationships as well as queries lacking apparent input.
In this section, we resolve these issues by introducing the notion of
\emph{query cardinality}.

We found it difficult to model these two relationships:
\begin{enumerate}[(i)]
\item \label{itm:employee-to-manager}
\emph{An employee may have a manager.}
\item \label{itm:department-to-employee}
\emph{A department is staffed by a number of employees.}
\end{enumerate}
We were also puzzled on how to express input-free queries such as:
\begin{enumerate}[(i)]
\setcounter{enumi}{2}
\item \label{itm:employee-set}
\emph{Show a list of all employees.}
\end{enumerate}

We could attempt to represent optional and plural output values as instances of
the container types
\begin{equation*}
    \Opt{A} \quad \text{and} \quad \Seq{A},
\end{equation*}
where the \emph{option} container $\Opt{A}$ holds zero or one value of type
$A$, and the \emph{sequence} container $\Seq{A}$ holds an ordered list of
values of type~$A$.  Using these containers,
relationships~\ref{itm:employee-to-manager}
and~\ref{itm:department-to-employee} could be expressed as primitive queries
with signatures
\begin{alignat*}{3}
    & \Manager && : \Emp && \to \Opt{\Emp}, \\
    & \Employee && : \Dept && \to \Seq{\Emp}.
\end{alignat*}
Moreover, we could guess the output of query~\ref{itm:employee-set}.  Indeed,
\emph{a list of all employees} can only mean $\Seq{\Emp}$.

To describe the input of query~\ref{itm:employee-set}, we introduce a
\emph{singleton} type
\begin{equation*}
    \Void.
\end{equation*}
The type $\Void$ has a unique inhabitant ($\top:\Void$), and because there is
no freedom in choosing a value of this type, it can designate input that can
never affect the result of a query.  Thus, we can
express~\ref{itm:employee-set} as a \emph{class} primitive
\begin{equation*}
    \Employee : \Void \to \Seq{\Emp}.
\end{equation*}
Note that even though both \ref{itm:department-to-employee} and
\ref{itm:employee-set} are denoted by the same name, we could always
distinguish them by their input type.

Unfortunately, although containers let us represent optional and plural output,
they do not compose well.  For example, it is tempting to express \emph{for a
given employee, find their manager's salary} as a composition
\begin{equation} \label{eq:manager-to-salary}
    \Manager\To\Salary, \tag{$\star$}
\end{equation}
or \emph{show the names of all employees} as
\begin{equation} \label{eq:employee-to-name}
    \Employee\To\Name. \tag{$\star\star$}
\end{equation}
However, if we look at the signatures of the components
\begin{alignat*}{6}
    & \Manager && : \Emp && \to \Opt{\Emp}, \quad && \Salary && : \Emp && \to \Int, \\
    & \Employee && : \Void && \to \Seq{\Emp}, \quad && \Name && : \Emp && \to \Text,
\end{alignat*}
we see that their intermediate domains do not agree, which means their
compositions are ill-formed.

To make the queries compose again, we should distinguish between the output
type of a query and its cardinality.\footnote{More precisely, we should
represent cardinalities as monads and queries as their Kleisli
arrows~\cite{Moggi1991}.} For example, we should say that
query~\ref{itm:employee-to-manager} is an optional query from $\Emp$ to $\Emp$,
\ref{itm:department-to-employee} is a plural query from $\Dept$ to $\Emp$, and
\ref{itm:employee-set} is a plural query from $\Void$ to $\Emp$.  Then, any two
queries should compose, regardless of their cardinalities, so long as they have
compatible intermediate types; furthermore, the least upper bound of their
cardinalities is the cardinality of their composition.

Specifically, given two queries
\begin{equation*}
    p : A \to \Wrap{M_1}{B}, \qquad q : B \to \Wrap{M_2}{C}
\end{equation*}
we first promote their output to a common cardinality
\begin{equation*}
    M = M_1 \sqcup M_2,
\end{equation*}
and then use the \emph{monadic composition} combinator
\begin{equation*}
    \placeholder\,\To\,\placeholder : (A \to \Wrap{M}{B},\; B \to \Wrap{M}{C}) \to (A \to \Wrap{M}{C})
\end{equation*}
to construct
\begin{equation*}
    p\,\To\,q : A \to \Wrap{M}{C}.
\end{equation*}
Using this rule, we can justify the queries (\ref{eq:manager-to-salary}) and
(\ref{eq:employee-to-name}) and give them signatures
\begin{alignat*}{3}
    & \Manager\To\Salary && : \Emp && \to \Opt{\Int}, \\
    & \Employee\To\Name && : \Void && \to \Seq{\Text}.
\end{alignat*}

Let us work out the details.  Query cardinalities are ordered with respect to
inclusions
\begin{equation*}
    A \sqsubseteq \Opt{A} \sqsubseteq \Seq{A},
\end{equation*}
which, using the notation for container instances
\begin{equation*}
    \bot,\; \optional{a} : \Opt{A}, \qquad [a_1,\ldots,a_n] : \Seq{A},
\end{equation*}
are defined by
\begin{alignat*}{5}
    & \; && \bot && : \Opt{A} && \longmapsto [\;] && : \Seq{A}, \\
    & a : A \longmapsto\ && \optional{a} && : \Opt{A} && \longmapsto [a] && : \Seq{A}.
\end{alignat*}
This order lets us, whenever necessary, promote any query $A\to\Wrap{M}{B}$ to
a query $A\to\Wrap{M'}{B}$ with a greater cardinality $M' \sqsupseteq M$.

Monadic composition for the option and sequence containers is well known.  For
optional queries
\begin{equation*}
    p : A \to \Opt{B}, \qquad q : B \to \Opt{C},
\end{equation*}
it is defined by
\begin{align*}
    & p\,\To\,q : A \to \Opt{C}, \\
    & p\,\To\,q : a \mapsto \begin{cases}
        \optional{c} & (p(a)=\optional{\,b},\; q(b)=\optional{c}), \\
        \bot & (\text{otherwise}).
    \end{cases}
\end{align*}
For plural queries
\begin{equation*}
    p : A \to \Seq{B}, \qquad q : B \to \Seq{C},
\end{equation*}
the sequence $(p\,\To\,q)(a)$ is calculated by applying $p$ to $a$
\begin{equation*}
    a \overset{p}{\longmapsto} [b_1, b_2, \ldots],
\end{equation*}
then applying $q$ to every element of $p(a)$
\begin{equation*}
    [b_1, b_2, \ldots]
    \overset{[q]}{\longmapsto}
    [[c^{1}_{1}, c^{2}_{1}, \ldots], [c^{1}_{2}, c^{2}_{2}, \ldots], \ldots],
\end{equation*}
and finally merging the nested sequences
\begin{equation*}
    [[c^{1}_{1}, c^{2}_{1}, \ldots], [c^{1}_{2}, c^{2}_{2}, \ldots], \ldots]
    \overset{\cancel{\,[\;]\,}}{\longmapsto}
    [c^{1}_{1}, c^{2}_{1}, \ldots, c^{1}_{2}, c^{2}_{2}, \ldots].
\end{equation*}

We are now ready to present the design of a combina\-tor-based query language.


\begin{figure}
    \centering
    \begin{tikzpicture}
        [
            > = stealth',
            shorten > = 1pt,
            node distance = 1.8cm and 0.5cm,
            set/.style = {
                draw, rectangle, thick, font=\sffamily,
                minimum width=1.5cm, minimum height=.75cm,
                text height=1.5ex, text depth=.25ex},
            map/.style = {font=\small\sffamily}
        ]
        \node [set] (Void) {Void};
        \node [set] (Dept) [below left=of Void] {Dept};
        \node [set] (Emp) [below right=of Void] {Emp};
        \node [set] (Text) [below=of Dept] {Text};
        \node [set] (Int) [below=of Emp] {Int};
        \draw [->] (Void) to [bend right=10] node [map, right] {department} (Dept);
        \draw [->] (Void) to [bend left=10] node [map, right] {employee} (Emp);
        \draw [->] (Dept) to [bend right=10] node [map, right] {name} (Text);
        \draw [->] (Dept) to [bend right=5] node [map, below] {employee} (Emp);
        \draw [->] (Emp) to [bend right=12] node [map, right] {\;name} (Text);
        \draw [->] (Emp) to [bend left=12] node [map, right] {\;position} (Text);
        \draw [->] (Emp) to [bend left=10] node [map, right] {salary} (Int);
        \draw [->] (Emp) to [bend right=5] node [map, above] {department} (Dept);
        \draw [->] (Emp) to [loop right,out=60,in=10,looseness=7] node [map, right] {manager} (Emp);
        \draw [->] (Emp) to [loop right,out=300,in=350,looseness=7] node [map, right] {subordinate} (Emp);
    \end{tikzpicture}
    \caption{Database schema in folded form}
    \label{fig:folded-form}
\end{figure}



\begin{description}
\item[Query model:]
A database query is characterized by its input type $A$, its output type $B$
and its cardinality $M$, and can be represented as a function of the form
\begin{equation*}
    p : A \to M\{B\},
\end{equation*}
where $M\{B\}$ is one of $B$, $\Opt{B}$ or $\Seq{B}$; the respective queries
are called singular, optional or plural.

\item[Primitives:]
The set of primitives includes classes
\begin{alignat*}{3}
    & \Department && : \Void && \to \Seq{\Dept}, \\
    & \Employee && : \Void && \to \Seq{\Emp},
\end{alignat*}
attributes
\begin{alignat*}{6}
    & \Name && : \Dept && \to \Text, \qquad
    && \Name && : \Emp && \to \Text, \\
    & \Position && : \Emp && \to \Text, \qquad
    && \Salary && : \Emp && \to \Int,
\end{alignat*}
and relationships
\begin{alignat*}{3}
    & \Department && : \Emp && \to \Dept, \\
    & \Employee && : \Dept && \to \Seq{\Emp}, \\
    & \Manager && : \Emp && \to \Opt{\Emp}, \\
    & \Subordinate && : \Emp && \to \Seq{\Emp}.
\end{alignat*}

\item[Combinators:]
The composition combinator sends two queries
\begin{equation*}
    p : A \to M_1\{B\}, \qquad
    q : B \to M_2\{C\}
\end{equation*}
to their composition
\begin{equation*}
    p\,\To\,q : A \to M\{C\} \qquad (M = M_1 \sqcup M_2).
\end{equation*}

Other common combinators are listed in Table~\ref{tab:common-combinators}.
They are described in the following sections.
\end{description}


\begin{figure}
    \centering
    \begin{forest}
        unfolded database
        [,void
            [$\Department$,map,plural
                [$\Name$,map,singular]
                [$\Employee$,map,plural
                    [$\Name$,map,singular]
                    [$\Position$,map,singular]
                    [$\Salary$,map,singular]
                    [$\Department$,map,singular
                        [,more]]
                    [$\Manager$,map,optional
                        [,more]]
                    [$\Subordinate$,map,plural
                        [,more]]
                    [,phantom]]]
            [$\Employee$,map,plural
                [$\Name$,map,singular]
                [$\Position$,map,singular]
                [$\Salary$,map,singular]
                [$\Department$,map,singular
                    [$\Name$,map,singular]
                    [$\Employee$,map,plural
                        [,more]]]
                [$\Manager$,map,optional
                    [,more]]
                [$\Subordinate$,map,plural
                    [,more]]]]
    \end{forest}
    \caption{Database schema in unfolded form}
    \label{fig:unfolded-form}
\end{figure}



Recall that we started with the schema graph in Figure~\ref{fig:sample-schema},
which gave us the original, incomplete set of primitives.  To reflect the
remaining primitives, we should add the $\Void$ node and the missing arcs (see
Figure~\ref{fig:folded-form}).  Furthermore, we will transform the schema graph
into an (infinite) tree by unfolding it starting from the $\Void$ node (see
Figure~\ref{fig:unfolded-form}).  The unfolded tree represents the functional
database in a universal hierarchical form.

