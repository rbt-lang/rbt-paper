
\section{Query Cardinality}
\label{sec:cardinality}

In Section~\ref{sec:introduction}, we suggested that any function defined on an
entity set can be seen as a query.  However, this naive model failed to
represent optional and plural relationships as well as queries lacking apparent
input.  In this section, we resolve these issues by introducing the notion of
\emph{query cardinality}.

We found it difficult to model these two relationships:

\begin{enumerate}[(i)]
\item \label{itm:employee-to-manager}
\emph{An employee may report to another employee, the manager.}

\item \label{itm:department-to-employee}
\emph{Any department has a number of associated employees.}
\end{enumerate}

We were also puzzled on how to express input-free queries such as:

\begin{enumerate}[(i)]
\setcounter{enumi}{2}
\item \label{itm:department-set}
\emph{Show a list of all departments.}
\end{enumerate}

In programming practice, optional and plural values are stored using container
types.  So let us define two parametric types: $\Opt{A}$ as a zero or one value
of type $A$ and $\Seq{A}$ as a type of finite $A$-valued sequences, that is
\begin{alignat*}{2}
    & \Opt{A} && = \{\, \bot \,\} + A, \\
    & \Seq{A} && = \{[a_1,\ldots,a_n] \mid a_1,\ldots,a_n:A,\; n=0,1,\ldots \}.
\end{alignat*}

With these, we can express relationships~(\ref{itm:employee-to-manager})
and~(\ref{itm:department-to-employee}) as primitives
\begin{alignat*}{3}
    & \Manager && : \Emp && \to \Opt{\Emp}, \\
    & \Employee && : \Dept && \to \Seq{\Emp}.
\end{alignat*}
The container structure over the query output is what we call the query
cardinality.

We can now guess the output type and cardinality of
query~(\ref{itm:department-set}).  Indeed, \emph{a list of all departments}
can only mean $\Seq{\Dept}$.

To describe the input of query~(\ref{itm:department-set}), we use a
\emph{singleton} type
\begin{equation*}
    \Void = \{\, \top \,\}.
\end{equation*}
Type $\Void$ has a unique inhabitant, and because there is no freedom in
choosing a value of this type, it can designate input that can never affect the
result of a query.  Thus, we can express~(\ref{itm:department-set}) with a
\emph{class} primitive
\begin{equation*}
    \Department : \Void \to \Seq{\Dept}
\end{equation*}
that produces all department entities.

Unfortunately, although containers let us represent optional and plural values,
they do not compose well.  For example, it is tempting to express a query that
\emph{for a given employee, finds their manager's salary} as a composition
\begin{equation} \label{eq:manager-to-salary}
    \Manager\To\Salary, \tag{$\star$}
\end{equation}
or a query that \emph{shows the names of all departments} as
\begin{equation} \label{eq:department-to-name}
    \Department\To\Name. \tag{$\star\star$}
\end{equation}
However, if we look at the signatures of the components
\begin{alignat*}{6}
    & \Manager && : \Emp && \to \Opt{\Emp}, \quad && \Salary && : \Emp && \to \Int, \\
    & \Department && : \Void && \to \Seq{\Dept}, \quad && \Name && : \Dept && \to \Text,
\end{alignat*}
we see that their intermediate domains do not agree, which means their
compositions are ill-formed.

To compose functions that carry extra structure with their output, we use
\emph{monadic composition} (\cite{Moggi1991}).  A mo\-nad is a parametric type
$M\{A\}$ equipped with a generic function $\keyword{pure}_A : A \to M\{A\}$ and
a monadic composition operator, which must satisfy certain coherence axioms.
Monadic composition of two functions
\begin{equation*}
    p : A \to M\{B\}, \qquad q : B \to M\{C\},
\end{equation*}
gives a function $p\,\To\,q$ with a signature of the same shape
\begin{equation*}
    p\,\To\,q : A \to M\{C\}.
\end{equation*}

Both $\Opt{A}$ and $\Seq{A}$ are known to be monads.  Let us confirm that they
admit monadic composition.

Monadic composition of
\begin{equation*}
    p : A \to \Opt{B}, \qquad q : B \to \Opt{C}
\end{equation*}
is just a composition of partial functions
\begin{equation*}
    p\,\To\,q : a \longmapsto \begin{cases}
        \bot & (p(a)=\bot\text{ or }q(p(a))=\bot), \\
        q(p(a)) & (\text{otherwise}).
    \end{cases}
\end{equation*}

To compose two sequence-valued functions
\begin{equation*}
    p : A \to \Seq{B}, \qquad q : B \to \Seq{C},
\end{equation*}
we first evaluate $p$ on the input $a$ of type $A$
\begin{equation*}
    a \overset{p}{\longmapsto} [b_1, b_2, \ldots],
\end{equation*}
then apply $q$ to every element of $p(a)$
\begin{equation*}
    [b_1, b_2, \ldots]
    \overset{[q]}{\longmapsto}
    [[c^{1}_{1}, c^{1}_{2}, \ldots], [c^{2}_{1}, c^{2}_{2}, \ldots], \ldots],
\end{equation*}
and finally flatten the nested sequences
\begin{equation*}
    [[c^{1}_{1}, c^{1}_{2}, \ldots], [c^{2}_{1}, c^{2}_{2}, \ldots], \ldots]
    \overset{\cancel{\,[\;]\,}}{\longmapsto}
    [c^{1}_{1}, c^{1}_{2}, \ldots, c^{2}_{1}, c^{2}_{2}, \ldots].
\end{equation*}
The result is the value $(p\,\To\,q)(a)$ of type $\Seq{C}$.

Monadic (and regular) composition rules give us a way to compose queries of the
same cardinality.  To compose queries with mixed cardinalities, we promote
their output to a smallest common monadic container using a chain of natural
embeddings
\begin{equation*}
    A \hookrightarrow \Opt{A} \hookrightarrow \Seq{A}.
\end{equation*}
These embeddings are defined by
\begin{alignat*}{5}
    & \; && \bot && : \Opt{A} && \longmapsto [\;] && : \Seq{A}, \\
    & a : A \longmapsto\ && a && : \Opt{A} && \longmapsto [a] && : \Seq{A}.
\end{alignat*}

We can now state the query composition rule as follows: find the largest
cardinality of the components; lift the output of all components to this
cardinality; use monadic composition.  For example, this rule gives queries
(\ref{eq:manager-to-salary}) and (\ref{eq:department-to-name}) signatures
\begin{alignat*}{3}
    & \Manager\To\Salary && : \Emp && \to \Opt{\Int}, \\
    & \Department\To\Name && : \Void && \to \Seq{\Text}.
\end{alignat*}

We are finally ready to present the design of a combina\-tor-based query
language.

\begin{description}
\item[Query model:]
A database query is a function of the form
\begin{equation*}
    p : A \to M\{B\}.
\end{equation*}
We call $A$ its input type, $B$ its output type, and $M$ its monadic
cardinality.  $M\{B\}$ could be one of $B$, $\Opt{B}$ or $\Seq{B}$ and
the respective queries are called singular, optional or plural.

\item[Primitives:]
The set of primitives includes class primitives
\begin{alignat*}{3}
    & \Department && : \Void && \to \Seq{\Dept}, \\
    & \Employee && : \Void && \to \Seq{\Dept},
\end{alignat*}
attributes
\begin{alignat*}{6}
    & \Name && : \Dept && \to \Text, \qquad
    && \Name && : \Emp && \to \Text, \\
    & \Position && : \Emp && \to \Text, \qquad
    && \Salary && : \Emp && \to \Int,
\end{alignat*}
and relationships
\begin{alignat*}{3}
    & \Department && : \Emp && \to \Dept, \\
    & \Employee && : \Dept && \to \Seq{\Emp}, \\
    & \Manager && : \Emp && \to \Opt{\Emp}, \\
    & \Subordinate && : \Emp && \to \Seq{\Emp}.
\end{alignat*}
Note that we allow multiple primitives to share the same name as long as they
are not defined with the same input type.

\item[Combinators:]
Binary composition combinator takes two queries
\begin{equation*}
    p : A \to M_1\{B\}, \qquad
    q : B \to M_2\{C\}
\end{equation*}
and produces a query of the form
\begin{equation*}
    p\,\To\,q : A \to M\{C\}.
\end{equation*}
Here, $M = M_1 \sqcup M_2$ is the smallest common cardinality of $M_1$ and
$M_2$.

More combinators will be described in the next sections.
\end{description}

This design suggests a new way to visually represent the structure of the
database.  Recall that we started with the schema graph in
Figure~\ref{fig:sample-schema}, which gave us the original, incomplete set of
primitives.  To reflect the primitives that were defined afterwards, we should
add the $\Void$ node and the missing arcs (see Figure~\ref{fig:folded-form}).
In this updated graph, any path corresponds to a composable chain of
primitives.

Furthermore, we can transform the schema graph into an (infinite) tree by
unfolding it starting from the $\Void$ node (see
Figure~\ref{fig:unfolded-form}).  In this form, each node corresponds to some
composition of primitives.  Figure~\ref{fig:unfolded-form} presents the
structure of the database in the hierarchical database model.


\begin{figure}
    \centering
    \begin{tikzpicture}
        [
            > = stealth',
            shorten > = 1pt,
            node distance = 1.8cm and 0.5cm,
            set/.style = {
                draw, rectangle, thick, font=\sffamily,
                minimum width=1.5cm, minimum height=.75cm,
                text height=1.5ex, text depth=.25ex},
            map/.style = {font=\small\sffamily}
        ]
        \node [set] (Void) {Void};
        \node [set] (Dept) [below left=of Void] {Dept};
        \node [set] (Emp) [below right=of Void] {Emp};
        \node [set] (Text) [below=of Dept] {Text};
        \node [set] (Int) [below=of Emp] {Int};
        \draw [->] (Void) to [bend right=10] node [map, right] {department} (Dept);
        \draw [->] (Void) to [bend left=10] node [map, right] {employee} (Emp);
        \draw [->] (Dept) to [bend right=10] node [map, right] {name} (Text);
        \draw [->] (Dept) to [bend right=5] node [map, below] {employee} (Emp);
        \draw [->] (Emp) to [bend right=12] node [map, right] {\;name} (Text);
        \draw [->] (Emp) to [bend left=12] node [map, right] {\;position} (Text);
        \draw [->] (Emp) to [bend left=10] node [map, right] {salary} (Int);
        \draw [->] (Emp) to [bend right=5] node [map, above] {department} (Dept);
        \draw [->] (Emp) to [loop right,out=60,in=10,looseness=7] node [map, right] {manager} (Emp);
        \draw [->] (Emp) to [loop right,out=300,in=350,looseness=7] node [map, right] {subordinate} (Emp);
    \end{tikzpicture}
    \caption{Database schema in folded form}
    \label{fig:folded-form}
\end{figure}




\begin{figure}
    \centering
    \begin{forest}
        unfolded database
        [,void
            [$\Department$,map,plural
                [$\Name$,map,singular]
                [$\Employee$,map,plural
                    [$\Name$,map,singular]
                    [$\Position$,map,singular]
                    [$\Salary$,map,singular]
                    [$\Department$,map,singular
                        [,more]]
                    [$\Manager$,map,optional
                        [,more]]
                    [$\Subordinate$,map,plural
                        [,more]]
                    [,phantom]]]
            [$\Employee$,map,plural
                [$\Name$,map,singular]
                [$\Position$,map,singular]
                [$\Salary$,map,singular]
                [$\Department$,map,singular
                    [$\Name$,map,singular]
                    [$\Employee$,map,plural
                        [,more]]]
                [$\Manager$,map,optional
                    [,more]]
                [$\Subordinate$,map,plural
                    [,more]]]]
    \end{forest}
    \caption{Database schema in unfolded form}
    \label{fig:unfolded-form}
\end{figure}



