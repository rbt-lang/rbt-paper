
\section{Quotient Classes}
\label{sec:quotients}

Previously, we demonstrated how to group and aggregate data---so long as the
structure of the data reflects the hierarchical form of the database.  In this
section, we show how to overcome this limitation.

In Figure~\ref{fig:unfolded-form}, the schema graph is unfolded into an
infinite tree, shaping the data into a hierarchical form.  A section of this
hierarchy could be extracted using the $\Select$ combinator.

\begin{demo}
    \label{ex:department-select-name-employee}
    Show all departments, and, for each department, list the associated
    employees.
    \begin{equation*}
        \Department\Apply\Select(\Name,\; \Employee)
    \end{equation*}
\end{demo}

What if we ask for \emph{positions} instead of \emph{departments}?

\begin{demo}
    \label{ex:employee-group-position}
    Show all positions, and, for each position, list the associated employees.
\end{demo}

Unlike the previous example, this query does not match the structure of the
database and, therefore, cannot be constructed as easily.  Indeed,
Example~\ref{ex:department-select-name-employee} is built from the primitives
\begin{alignat*}{3}
    & \Department && : \Void && \to \Seq{\Dept}, \\
    & \Name && : \Dept && \to \Text, \\
    & \Employee && : \Dept && \to \Seq{\Emp}.
\end{alignat*}
To construct Example~\ref{ex:employee-group-position} in a similar fashion, we
need a hypothetical class $\Pos$ of \emph{position} entities and a set of
queries with the corresponding signatures
\begin{equation}
    \label{eq:group-components}
    \begin{alignedat}{2}
        & \Void && \to \Seq{\Pos}, \\
        & \Pos && \to \Text, \\
        & \Pos && \to \Seq{\Emp}.
    \end{alignedat} \tag{$\star\star\star$}
\end{equation}
However, there is no built-in class of \emph{position} entities and we only
have the following primitives available:
\begin{alignat*}{3}
    & \Employee && : \Void && \to \Seq{\Emp}, \\
    & \Position && : \Emp && \to \Text.
\end{alignat*}

To make a ``virtual'' entity class from all distinct values of an attribute and
inject this class into the database structure, we use the $\Group$ combinator.
For example (see Figure~\ref{fig:group-combinator}), a list of \emph{all
distinct employee positions} can be produced with the query
\begin{equation*}
    \Employee\Apply\Group(\Position) : \Void \to \Seq{\Pos}.
\end{equation*}
The virtual $\Pos$ class comes with the primitives
\begin{alignat*}{3}
    & \Position && : \Pos && \to \Text, \\
    & \Employee && : \Pos && \to \Seq{\Emp},
\end{alignat*}
which, \emph{given a position entity, produce respectively the position name
and a list of associated employees}.
This gives us all the query components (see~(\ref{eq:group-components}) above)
needed to complete the example.


\begin{figure}
    \centering
    \begin{forest}
        unfolded database
        [,phantom
            [,phantom
                [,void
                    [$\Employee$,map,plural,selected,name=src employee
                        [$\Name$,map,singular]
                        [$\Position$,map,singular,selected,selected edge]
                        [$\Salary$,map,singular,and more]
                        [,phantom]]]]
            [,void
                [$\begin{aligned}&\Employee\\&\Apply\Group(\Position)\end{aligned}$,box,plural,selected,name=tgt group
                    [$\Position$,map,singular,selected,selected edge]
                    [$\Employee$,map,plural,selected,selected edge
                        [$\Name$,map,singular]
                        [$\Position$,map,singular]
                        [$\Salary$,map,singular,and more]]]]]
        \draw[traverse] (src employee.south east) +(0,-0.25cm) to [bend right=10] (tgt group);
    \end{forest}
    \caption{Action of the $\Group$ combinator}
    \label{fig:group-combinator}
\end{figure}



\addtocounter{demo}{-1}
\begin{demo}
    Show all positions, and, for each position, list the associated employees.
    \begin{align*}
        & \Employee \\
        & \Apply\Group(\Position) \\
        & \Apply\Select(\Position,\; \Employee)
    \end{align*}
\end{demo}

The query
\begin{equation*}
    \Employee\Apply\Group(\Position)
\end{equation*}
correlates all distinct values emitted by $\Employee\To\Position$ with the
respective $\Employee$ entities and packs them together into the records of
type
\begin{equation*}
    \Pos \equiv \Tuple{\Position : \Text,\; \Employee : \Seq{\Emp}}.
\end{equation*}
We call $\Pos$ a \emph{quotient class} and denote it by
\begin{equation*}
    \Quotient{\Emp}{\Position}.
\end{equation*}

Once the database hierarchy is rearranged to include the class $\Pos$, we can
answer any questions about position entities.

\begin{demo}
    \label{ex:employee-group-position-etc}
    In the Police department, show all positions with the number of employees
    and the top salary.
    \begin{alignat*}{2}
        & \Employee\hidewidth && \\
        & \Apply\Filter(\Department\To\Name = \textliteral{POLICE})\hspace{-15em}&&\hspace{12em} \\
        & \Apply\Group(\Position)\hidewidth && \\
        & \Apply\Select(&& \Position, \\
        & && \Count(\Employee), \\
        & && \Max(\Employee\To\Salary))
    \end{alignat*}
\end{demo}

Here, for each position in the Police department, we determine two calculated
attributes, the number of employees and the top salary:
\begin{alignat*}{3}
    & \Count(\Employee) && : \Quotient{\Emp}{\Position} && \to \Int, \\
    & \Max(\Employee\To\Salary) && : \Quotient{\Emp}{\Position} && \to \Opt{\Int}.
\end{alignat*}

\begin{demo}
    \label{ex:nested-group}
    Arrange employees into a hierarchy: first by position, then by department.
    \begin{alignat*}{2}
        & \Employee\hidewidth && \\
        & \Apply\Group(\Position)\hidewidth && \\
        & \Apply\Select(&& \Position, \\
        & && \Employee \\
        & && \Apply\Group(\Department) \\
        & && \Apply\Select(\Department\To\Name,\; \Employee))
    \end{alignat*}
\end{demo}

Nested $\Group$ combinators can construct a hierarchical output of an arbitrary
form.  In this example, we rebuild the database hierarchy to place positions on
top, then departments, and then employees.  Notably, the nested $\Group$
expression has a signature
\begin{multline*}
    \Employee\Apply\Group(\Department) : \\
    \Quotient{\Emp}{\Position} \to \Seq{\Quotient{\Emp}{\Department}}.
\end{multline*}

\begin{demo}
    \label{ex:unique-department}
    Show all positions available in more than one department, and, for each
    position, list the respective departments.
    \begin{alignat*}{2}
        & \Employee\hidewidth &&  \\
        & \Apply\Group(\Position)\hidewidth && \\
        & \Apply\Define(&& \Department\As \\
        & && \qquad \Unique(\Employee\To\Department)) \\
        & \Apply\Filter(\Count(\Department) > 1)\hidewidth && \\
        & \Apply\Select(\Position,\; \Department\To\Name)\hidewidth &&
    \end{alignat*}
\end{demo}

This example uses the $\Unique$ combinator to find all distinct values in a
list of departments.  The $\Unique$ combinator can be expressed via $\Group$ by
forgetting the plural component of the quotient class.  In this example,
$\Unique(\Employee\To\allowbreak\Department)$ is equivalent to
\begin{equation*}
    \Employee\Apply\Group(\Department)\To\Department.
\end{equation*}

\begin{demo}
    \label{ex:employee-group-level}
    How many employees at each level of the organization chart?
    \begin{align*}
        & \Employee \\
        & \Apply\Group(\Level\As\Count(\Connect(\Manager))) \\
        & \Apply\Select(\Level,\; \Count(\Employee))
    \end{align*}
\end{demo}

This example demonstrates that $\Group$ could be applied to a calculated
attribute such as
\begin{equation*}
    \Count(\Connect(\Manager)) : \Emp \to \Int.
\end{equation*}

The $\Group$ combinator could also be applied to more than one attribute, which
can be used for summarizing data along several dimensions.  When the summary
data has to include subtotals and totals, we replace $\Group$ with $\Rollup$.

\begin{demo}
    \label{ex:rollup}
    Show the average salary by department and position, with subtotals for each
    department and the grand total.
    \begin{alignat*}{2}
        & \Employee\hidewidth && \\
        & \Apply\Rollup(\Department,\; \Position)\hidewidth && \\
        & \Apply\Select(&& \Department, \\
        & && \Position, \\
        & && \Mean(\Employee\To\Salary))
    \end{alignat*}
\end{demo}

In this example, the query
\begin{equation*}
    \Employee\Apply\Rollup(\Department,\; \Position)
\end{equation*}
produces a sequence of records of type
\begin{multline*}
    \Quotient{\Emp}{\Department_{\bot},\Position_{\bot}} \equiv \\
    \begin{alignedat}{3}
        & \Tuple{&& \Department && : \Opt{\Dept}, \\
        & && \Position && : \Opt{\Text}, \\
        & && \Employee && : \Seq{\Emp}}.
    \end{alignedat}
\end{multline*}
In addition to the records that would be generated by $\Group$, $\Rollup$ emits
a ``subtotal'' record per each department, with the $\Position$ field set to
$\bot$, and one ``grand total'' record, with both $\Department$ and $\Position$
set to $\bot$.

