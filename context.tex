
\section{Query Context}
\label{sec:context}

In this section, we extend the query model to support \emph{context-aware}
queries: parameterized queries and queries with window functions.

\begin{demo}
    \label{ex:parameters}
    Show all employees in the given department $\DEPARTMENT$ with the salary
    higher than $\SALARY$, where
    \begin{equation*}
        \DEPARTMENT=\textliteral{POLICE}, \quad \SALARY=150000.
    \end{equation*}
    \begin{align*}
        & \Employee \\
        & \Apply\Filter(\Department\To\Name = \DEPARTMENT \mathbin{\&} \Salary > \SALARY) \\
        & \Apply\Given(\DEPARTMENT \As \textliteral{POLICE},\; \SALARY \As 150000)
    \end{align*}
\end{demo}

Practical database queries often depend upon \emph{query parameters}, which
collectively form the query \emph{environment}.  The environment is represented
by a container, such as
\begin{equation*}
    \Env{\DEPARTMENT:\Text,\SALARY:\Int}{A} \equiv \Tuple{A,\; \Tuple{\DEPARTMENT:\Text,\; \SALARY:\Int}},
\end{equation*}
that encapsulates both the regular input value and the values of the
parameters.  The parameters can be extracted from the environment with the
primitives
\begin{equation*}
    \DEPARTMENT : \Env{\DEPARTMENT:\Text}{A} \to \Text, \qquad
    \SALARY : \Env{\SALARY:\Int}{A} \to \Int.
\end{equation*}

The query environment is populated using the combinator $\Given$.  In this
example, the first argument of $\Given$ is a parameterized query
\begin{multline*}
    \Employee \\
    \shoveleft{\Apply\Filter(\Department\To\Name = \DEPARTMENT \mathbin{\&} \Salary > \SALARY) : } \\
    \Env{\DEPARTMENT:\Text,\SALARY:\Int}{\Void} \to \Seq{\Emp}.
\end{multline*}
The other two arguments are the constant queries
\begin{equation*}
    \textliteral{POLICE} : \Void \to \Text, \qquad
    150000 : \Void \to \Int
\end{equation*}
that specify the values of the parameters.  The combined query does \emph{not}
depend upon the parameters, and, hence, has a signature
\begin{equation*}
    \Void \to \Seq{\Emp}.
\end{equation*}

In general, $\Given$ takes a parameterized query
\begin{equation*}
    p : \Env{x_1:T_1,\ldots,x_n:T_n}{A} \to \Wrap{M}{B},
\end{equation*}
$n$ queries that evaluate the parameters
\begin{equation*}
    q_1 : A \to T_1,\quad \ldots,\quad q_n : A \to T_n
\end{equation*}
and combines them into a context-free query
\begin{align*}
    & \Given(p,\; q_1,\; \ldots,\; q_n) : A \to \Wrap{M}{B}, \\
    & \Given(p,\; q_1,\; \ldots,\; q_n) = a \mapsto p(\Tuple{a, \Tuple{q_1(a), \ldots, q_n(a)}}).
\end{align*}

\begin{demo}
    \label{ex:higher-than-average-salary}
    Which employees have higher than average salary?
    \begin{align*}
        & \Employee \\
        & \Apply\Filter(\Salary > \MEANSALARY) \\
        & \Apply\Given(\MEANSALARY \As \Mean(\Employee\To\Salary))
    \end{align*}
\end{demo}

This example uses the query environment to pass information between different
scopes.  The parameter $\MEANSALARY$ is calculated in the scope of $\Void$ by
the query
\begin{equation*}
    \Mean(\Employee\To\Salary) : \Void \to \Opt{\Num}
\end{equation*}
and is extracted in the scope of $\Emp$ by the primitive
\begin{equation*}
    \MEANSALARY : \Env{\MEANSALARY:\Opt{\Num}}{\Emp} \to \Opt{\Num}.
\end{equation*}

The query environment is one example of a \emph{query context}, a comonadic
container wrapping the query input.  It could be shown that the environment is
compatible with query composition (cf.~Section~\ref{sec:cardinality}), which
permits us to incorporate it into the query model.

Another example of a query context is the \emph{input flow}, a container of
every input value seen by the query.  We denote this context type by $\Rel{A}$
and its values by
\begin{equation*}
    [a_1,\;\ldots,\selected{a_j},\;\ldots,\;a_n] : \Rel{A},
\end{equation*}
where $a_j$ is the current input value, $a_1,\;\ldots,\;a_{j-1}$ are the values
seen in the past, and $a_{j+1},\;\ldots,\;a_n$ are the values to be seen in the
future.  The input flow can be used for an alternative implementation of
Example~\ref{ex:higher-than-average-salary}.

\begin{demobis}{ex:higher-than-average-salary}
    Which employees have higher than average salary?
    \begin{equation*}
        \Employee\Apply\Filter(\Salary > \Mean(\Around\To\Salary))
    \end{equation*}
\end{demobis}

When a query needs to relate each value in a dataset to the dataset as a whole,
we can use the plural primitive $\Around$, which materializes its input flow as
a sequence:
\begin{multline*}
    \Around : \Rel{A} \to \Seq{A} \\
    \shoveleft{\Around = [a_1,\;\ldots,\selected{a_j},\;\ldots,\;a_n]} \\
    \mapsto [a_1,\;\ldots,\;a_j,\;\ldots,\;a_n].
\end{multline*}
In this example, $\Around$ produces, \emph{for a given employee, a list of all
employees}.  By composing it with $\Salary$, we get, \emph{for a given
employee, a list of all salaries}
\begin{equation*}
    \Around\To\Salary : \Rel{\Emp} \to \Seq{\Int},
\end{equation*}
which lets us determine the average salary
\begin{equation*}
    \Mean(\Around\To\Salary) : \Rel{\Emp} \to \Opt{\Num}
\end{equation*}
without ever leaving the current scope.

\begin{demo}
    In the Police department, show employees whose salary is higher than the
    average for their position.
    \begin{align*}
        & \Employee \\
        & \Apply\Filter(\Department\To\Name=\textliteral{POLICE}) \\
        & \Apply\Filter(\Salary > \Mean(\Around(\Position)\To\Salary))
    \end{align*}
\end{demo}

Here, each employee is matched with other employees having the same position
using a variant of $\Around$:
\begin{multline*}
    \Around : (A \to B) \to (\Rel{A} \to \Seq{A}) \\
    \shoveleft{\Around(q) = [a_1,\;\ldots,\selected{a_j},\;\ldots,\;a_n]} \\
    \mapsto [\,a_i \mid q(a_i)=q(a_j)\,].
\end{multline*}
Note the use of two separate $\Filter$ combinators.  If we merge them into one,
$\Around(\Position)$ would list employees with the same position \emph{across
all departments}.

We can exploit the input flow to calculate running aggregates.

\begin{demo}
    Show a numbered list of employees and their salaries along with the running
    total.
    \begin{alignat*}{3}
        & \Employee\hidewidth && && \\
        & \Apply\Select(&& \No && \As \Count(\Before), \\
        & && \Name,\hidewidth && \\
        & && \Salary,\hidewidth &&  \\
        & && \Total && \As \Sum(\Before\To\Salary))
    \end{alignat*}
\end{demo}

The primitive $\Before$ exposes its input flow up to and including the current
input value:
\begin{align*}
    & \Before : \Rel{A} \to \Seq{A} \\
    & \Before = [a_1,\;\ldots,\selected{a_j},\;\ldots,\;a_n] \mapsto [a_1,\;\ldots,\;a_j].
\end{align*}
Using $\Before$, we can enumerate the rows in the output
\begin{equation*}
    \Count(\Before) : \Rel{\Emp} \to \Int
\end{equation*}
as well as calculate the running sum of salaries
\begin{equation*}
    \Sum(\Before\To\Salary) : \Rel{\Emp} \to \Int.
\end{equation*}

\begin{demo}
    For each department, show employee salaries along with the running total;
    the total should be reset at the department boundary.
    \begin{alignat*}{2}
        & \Department\hidewidth && \\
        & \Apply\Select(&& \Name, \\
        & && \Employee \\
        & && \Apply\Select(\Name,\; \Salary,\; \Sum(\Before\To\Salary)) \\
        & && \Apply\Frame)
    \end{alignat*}
\end{demo}

The input flow propagates through composition, so that a query executed
within
\begin{equation*}
    \Department\To\Employee : \Void \to \Seq{\Emp}
\end{equation*}
will see the input flow containing all the employees in all departments.  To
reset the input flow at a certain boundary, we use the combinator
\begin{equation*}
    \Frame : (\Rel{A} \to \Wrap{M}{B}) \to (A \to \Wrap{M}{B}).
\end{equation*}

