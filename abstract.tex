
\begin{abstract}
    We introduce Rabbit, a combinator-based query language.  Rabbit is designed
    for data analysts and other accidental programmers to let them query
    complex structured data.

    Starting with the functional database model and using the methods of
    categorical semantics of computations, we develop denotational semantics of
    database queries.  In Rabbit, a query is modeled as a mapping with certain
    input type, output type and cardinality, where query cardinality is a
    monadic container holding the query output.  In this model, monadic
    composition can be used to navigate the database, while other \emph{query
    combinators} can aggregate, filter, sort and paginate data, construct
    compound data, connect self-referential data, and reorganize data with
    grouping and database cube operations.  A context-aware query model, with
    the input context represented as a como\-nadic container, can express query
    parameters and window functions.  Rabbit semantics enables pipeline
    notation, encouraging its users to construct database queries as a series
    of distinct steps, each individually crafted and tested.  We believe that
    for data analytics, Rabbit presents a superior alternative to SQL.
\end{abstract}

