
\begin{abstract}
    We introduce Rabbit, a combinator-based query language.  Rabbit is designed
    to let data analysts and other \emph{accidental programmers} query complex
    structured data.

    We combine the functional data model and the categorical semantics of
    computations to develop denotational semantics of database queries.  In
    Rabbit, a query is modeled as a Kleisli arrow for a monadic container
    determined by the query cardinality.  In this model, monadic composition
    can be used to navigate the database, while other \emph{query combinators}
    can aggregate, filter, sort and paginate data; construct compound data;
    connect self-referential data; and reorganize data with grouping and data
    cube operations.  A context-aware query model, with the input context
    represented as a como\-nadic container, can express query parameters and
    window functions.  Rabbit semantics enables pipeline notation, encouraging
    its users to construct database queries as a series of distinct steps, each
    individually crafted and tested.  We believe that Rabbit can serve as a
    practical tool for data analytics.
\end{abstract}

