
\section{Introduction}
\label{sec:introduction}

\emph{Combinators} are a popular approach to the design of compositional
domain-specific languages (DSLs). This approach views a DSL as an algebra of
self-con\-tained processing blocks, which either come from a set of predefined
atomic \emph{primitives} or are constructed from other blocks using block
combinators.

The combinator approach gives us a roadmap to design a database query language:

\begin{itemize}
\item
define the model of database queries;

\item
describe the set of primitive queries;

\item
describe the combinators for making composite queries.
\end{itemize}

To elaborate on this idea, we need some sample structured data.  Throughout
this paper, we use a simple database that contains just two classes of
entities: \emph{departments} and \emph{employees}.  Each department entity has
one attribute: \emph{name}.  Each employee entity has three attributes:
\emph{name}, \emph{position} and \emph{salary}.  Each employee belongs to a
department.  An employee may have a \emph{manager}, who is also an employee.

In Figure~\ref{fig:sample-schema}, the structure of the sample database is
visualized as a directed graph, with attributes and relationships (arcs)
connecting entity classes and attribute types (graph nodes).  This diagram may
suggest that we view attributes and relationships as functions with the given
types of input and output, for example
\begin{alignat*}{3}
    & \Department && : \Emp && \to \Dept, \\
    & \Name && : \Dept && \to \Text.
\end{alignat*}
This is known as the functional database model \cite{Kerschberg1976,
Sibley1977}.


\begin{figure}
    \centering
    \begin{tikzpicture}[entity diagram]
        \node [set] (Dept) {Dept};
        \node [set] (Emp) [right=of Dept] {Emp};
        \node [set] (Text) [below=of Dept] {Text};
        \node [set] (Int) [below=of Emp] {Int};
        \draw [->] (Dept) to [bend right=15] node [map, right] {name} (Text);
        \draw [->] (Emp) to [bend right=15] node [map, right] {\;name} (Text);
        \draw [->] (Emp) to [bend left=15] node [map, right] {\;position} (Text);
        \draw [->] (Emp) to [bend left=15] node [map, right] {salary} (Int);
        \draw [->] (Emp) to node [map, above] {department} (Dept);
        \draw [->] (Emp) to [loop right] node [map, right] {manager} (Emp);
    \end{tikzpicture}
    \caption{Sample database}
    \label{fig:sample-schema}
\end{figure}



This model provides us with a starting point on our combinator roadmap.
Indeed, a database query could be seen as a function; then, a set of primitive
queries is formed by all the attributes and relationships, while function
composition becomes a binary query combinator.  With these considerations, we
can write our first composite query.

\begin{demo}
    \label{ex:composite-query}
    Given an employee entity, show the name of their department.
    \begin{equation*}
        \Department\To\Name: \Emp \to \Text
    \end{equation*}
\end{demo}

In this example, $\Department\To\Name$ is a query written in Rabbit notation,
and $\Emp \to \Text$ is its signature.  The period (``$\To$'') denotes the
composition combinator, which is a polymorphic binary operator with a signature
\begin{equation*}
        \placeholder\,\To\,\placeholder : (A \to B,\; B \to C) \to (A \to C).
\end{equation*}

Even though this query model can express one database query, it does not seem
to be powerful enough to become the foundation of a query language.  What is
this model missing?

First, it is awkward that a query always demands an input.  It means that we
cannot express an input-free query like \emph{show a list of all
employees}.\footnote{We italicize \emph{business questions} that specify
$\keyword{database}$ $\keyword{queries}$.}

Further, although the relationships are bidirectional, the model only covers
one of their directions.  Indeed, we chose to represent the relationship
between departments and employees as a primitive with input $\Emp$ and output
$\Dept$.  However, we may just as well be interested in finding, \emph{for any
given department, the corresponding list of employees}.  It would be natural to
add a primitive for the opposite direction, but it cannot be encoded as a
function because its signature $\Dept \to \Emp$ would incorrectly imply that
there is exactly one employee per department.  Thus, the query model is unable
to express multivalued or \emph{plural} relationships.

The model also fails to capture the semantics of \emph{optional} attributes and
relationships.  Such is the relationship between employees and their managers,
which, according to Figure~\ref{fig:sample-schema}, should be encoded by a
primitive with signature $\Emp \to \Emp$.  But this signature implies that
every employee must have a manager, which is untrue.  Apparently, a pure
functional model is too restrictive to express the variety of relationships
between database entities.

This paper shows how to complete this query model and build a query language on
top of it.  It is organized as follows.

In Section~\ref{sec:cardinality}, we show how to represent optional and plural
relationships using the notion of query cardinality, which, following the
approach of categorical semantics of computations~\cite{Moggi1991}, determines
the monadic container for the query output.  This lets us establish a
compositional model of database queries.

In Section~\ref{sec:combinators}, we show how common data operations can be
expressed as query combinators.  Specifically, we describe combinators that
extract, aggregate, filter, sort and paginate data; construct compound data; and
connect self-referential data.

In Section~\ref{sec:quotients}, we show how grouping and data cube operations
can be implemented as combinators that reorganize the intrinsic hierarchical
structure of the database.

In Section~\ref{sec:context}, using the approach to the semantics of dataflow
programming~\cite{Uustalu2005}, we extend the query model to include a
comonadic query context, which allows us to express query parameters and window
functions.

In Section~\ref{sec:conclusion}, we summarize the query model and briefly
discuss some related work.

