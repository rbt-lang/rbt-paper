
\section{Conclusion and Related Work}
\label{sec:conclusion}

We designed a combinator-based database query language and, using the
framework of (bi-)Kleisli arrows \cite{Moggi1991, Uustalu2006}, described its
denotational semantics.

Specifically, we started with some base category containing entity classes
\begin{equation*}
    \Dept,\; \Emp,\; \ldots,
\end{equation*}
simple value types
\begin{equation*}
    \Void,\; \Bool,\; \Int,\; \Text,\; \ldots,
\end{equation*}
and container types
\begin{equation*}
    \Tuple{\ldots},\; \Opt{\ldots},\; \Seq{\ldots}, \ldots.
\end{equation*}
We bootstrapped the query model by assuming that a query with input of type $A$
and output of type $B$ can be expressed as an arrow
\begin{equation*}
    A \to B.
\end{equation*}
To model query cardinality, we wrapped the output type in a monadic container:
\begin{equation*}
    A \to \Wrap{M}{B}.
\end{equation*}
Query cardinalities form a family
$\mathcal{M}=\{\,\keyword{1},\;\keyword{Opt},\;\keyword{Seq}\,\}$ of monads
equipped with a semilattice structure: for any
for any $M_1, M_2 \in \mathcal{M}$, there exists
$M_1 \sqcup M_2 \in \mathcal{M}$ with natural injections
\begin{equation*}
    \Wrap{M_1}{A} \rightarrow \Wrap{(M_1 \sqcup M_2)}{A} \leftarrow \Wrap{M_2}{A}.
\end{equation*}
In order to model query context, we wrapped the input type in a dual structure,
a comonadic container:
\begin{equation*}
    \Wrap{W}{A} \to \Wrap{M}{B}.
\end{equation*}
Dually, query context is represented by a semilattice
$\mathcal{W}=\{\,\keyword{1},\;\keyword{Rel},\;\keyword{Env}_{\Int},\;\ldots\,\}$
of comonads: for any $W_1, W_2 \in \mathcal{W}$, there exists
$W_1 \sqcap W_2 \in \mathcal{W}$ with natural projections
\begin{equation*}
    \Wrap{W_1}{A} \leftarrow (W_1 \sqcap W_2)\{A\} \rightarrow \Wrap{W_2}{A}.
\end{equation*}
Moreover, for any monad $M\in\mathcal{M}$ and comonad $W\in\mathcal{W}$, there
exists a natural transformation called a distributive law:
\begin{equation*}
    W\{M\{A\}\} \to M\{W\{A\}\}.
\end{equation*}
Composition of queries
\begin{equation*}
    p : W_1\{A\} \to M_1\{B\}, \quad q : W_2\{B\} \to M_2\{C\}
\end{equation*}
is a query of the form
\begin{multline*}
    p\,\To\,q : W\{A\} \to M\{B\} \\ (W = W_1 \sqcap W_2,\; M = M_1 \sqcup M_2),
\end{multline*}
which is constructed using the lattice structures of $\mathcal{M}$ and
$\mathcal{W}$, compositional properties of monads and comonads, and the
distributive law for $M$ and $W$.

