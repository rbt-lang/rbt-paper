
\section{Conclusion and Related Work}
\label{sec:conclusion}

In this paper, we introduce a combinator-based query language, \emph{Rabbit},
and, using the framework of (co)monads and (bi-)Kleisli arrows \cite{Moggi1991,
Uustalu2005}, describe the denotation of database queries.

The functional database model gives us the underlying category of serializable
data.  It consists of entity classes such as $\Dept$ and $\Emp$; simple value
types $\Void$, $\Bool$, $\Int$, $\Text$ and others; and their composites
$\Tuple{\ldots}$, $\Opt{\ldots}$, and $\Seq{\ldots}$.  We bootstrap the query
model by assuming that a query with input of type $A$ and output of type $B$
can be expressed as an arrow
\begin{equation*}
    A \to B.
\end{equation*}
To model optional and plural queries, we wrap their output in a monadic
container and represent them as Kleisli arrows
\begin{equation*}
    A \to \Wrap{M}{B}.
\end{equation*}
The containers form a family $\mathcal{M}$ of monads equipped with a
join-semilattice structure: for any $M_1, M_2 \in \mathcal{M}$, there exists
$M_1 \sqcup M_2 \in \mathcal{M}$ with natural injections
\begin{equation*}
    \Wrap{M_1}{A} \rightarrow \Wrap{(M_1 \sqcup M_2)}{A} \leftarrow \Wrap{M_2}{A}.
\end{equation*}
To represent query parameters and the input flow, we wrap the query input in a
comonadic container, expressing context-aware queries as bi-Kleisli arrows
\begin{equation*}
    \Wrap{W}{A} \to \Wrap{M}{B}.
\end{equation*}
Dually, the comonadic containers form a meet-semi\-lat\-tice $\mathcal{W}$ of
comonads: for any $W_1, W_2 \in \mathcal{W}$, there exists
$W_1 \sqcap W_2 \in \mathcal{W}$ with natural projections
\begin{equation*}
    \Wrap{W_1}{A} \leftarrow (W_1 \sqcap W_2)\{A\} \rightarrow \Wrap{W_2}{A}.
\end{equation*}
Moreover, for any monad $M\in\mathcal{M}$ and comonad $W\in\mathcal{W}$, there
should exist a distributive law
\begin{equation*}
    \Wrap{W}{\Wrap{M}{A}} \to \Wrap{M}{\Wrap{W}{A}}.
\end{equation*}
Then, the composition of queries
\begin{equation*}
    p : \Wrap{W_1}{A} \to \Wrap{M_1}{B}, \quad q : \Wrap{W_2}{B} \to \Wrap{M_2}{C}
\end{equation*}
could be defined as a query of the form
\begin{multline*}
    p\,\To\,q : \Wrap{W}{A} \to \Wrap{M}{C} \\ (W = W_1 \sqcap W_2,\; M = M_1 \sqcup M_2)
\end{multline*}
constructed using the lattice structures of $\mathcal{M}$ and $\mathcal{W}$,
compositional properties of monads and comonads, and the distributive law for
$M$ and $W$.

Rabbit has its roots in the authors' work on a URL-based query
language~\cite{Evans2007}, which provided a navigational interface to SQL
databases.  While looking for a way to formally specify this language, we
arrived at the combinator-based query model.

Early on, we adopted the navigational approach of XPath~\cite{Clark1999}, which
led us to represent the schema as a rooted graph (e.g.,
Figure~\ref{fig:folded-form}) and queries as paths in this graph.  We
recognized that each graph arc has some cardinality, and, consequently, so does
each path.  Next came the realization that, for any dataset, the dataset values
are all related to each other, and this relationship can be denoted as a plural
self-referential arc $\Around$.  We discovered that the rule for composing
$\Around$ with other plural arcs is exactly the distributive law for the
$\type{Rel}$ comonad over the $\type{Seq}$ monad, which pushed us to model
database queries as Kleisli arrows.

Monads and their Kleisli arrows came to be a standard tool in denotational
semantics after Moggi~\cite{Moggi1991} used them to define a generic
compositional model of computations.  By varying the choice of monad, he
expressed partiality, exceptions, input-output, and other computational
effects.  Uustalu and Vene~\cite{Uustalu2005} used a dual model of comonads and
co-Kleisli arrows to describe semantics of dataflow programming.  They also
discussed distributive laws of a comonad over a monad.  In the context of
databases, Spivak~\cite{Spivak2012} suggested using monads to encode data with
complex structure.  Monad comprehensions~\cite{Trinder1989, Buneman1994} form
the core of query interfaces such as Kleisli~\cite{Wong2000} and
LINQ~\cite{Meijer2006}.  In contrast with Rabbit, which is based on Kleisli
arrows and monadic composition, these interfaces are modeled around monadic
containers and the monadic \emph{bind} operator.

The graph representation of the database schema is a variation of the
functional database model~\cite{Kerschberg1976, Sibley1977}, which gave rise to
a number of query languages: FQL~\cite{Buneman1979}, DAPLEX~\cite{Shipman1981},
GENESIS~\cite{Batory1988}, Kleisli~\cite{Wong2000} and others;
see~\cite{Gray2004} for a comprehensive survey.  Among them, FQL and its
derivatives are remarkably close to Rabbit---Example~\ref{ex:composite-query}
is a valid query in both.  The key difference is that we interpret the period
(``$\To$'') as a composition of Kleisli arrows, which implies, for instance,
that we cannot define $\Count$ as $\Seq{A}\to\Int$ and write
$\Employee\To\Count$ for \emph{the number of employees}.  Instead, we have to
accept $\Count$ as a query combinator.

Combinators are higher-order functions that serve to construct expressions
without bound variables.  They were introduced as the building blocks of
mathematical logic~\cite{Schoenfinkel1924, Curry1930}, from where they migrated
to programming practice, becoming a popular tool for constructing DSLs;
examples are found in diverse domains such as parsers~\cite{Wadler1985,
Hutton1996}, reactive animation~\cite{Elliott1997}, financial
contracts~\cite{Jones2000}, and the view-update problem~\cite{Foster2005}.

Although a few combinator-based query models have been
proposed~\cite{Buneman1979, Bossi1984, Batory1988, Erwig1991, Cherniack1996},
it is generally accepted that ``combinator-style languages are difficult for
users to master and thus ill-suited as query languages''~\cite{Cherniack1996}.
We have to disagree.  The syntax of a combinator-based DSL directly mirrors its
semantics, making it an \emph{executable specification}.  This is an attractive
property for a language oriented towards domain experts---if the semantics does
not contradict the experts' intuition.

In Rabbit, the intuition relies upon the hierarchical data model, which is
simple, familiar and prolific.  For querying purposes, we view a functional
database as a universal hierarchical document obtained by unfolding the
database schema into a potentially infinite schema tree (e.g.,
Figure~\ref{fig:unfolded-form}).  We learned this technique from concurrency
theory, where static ``system'' models are unfolded into runtime ``behavior''
models~\cite{Nielsen1994}.  This technique has also been used to specify an
adjunction between the network and hierarchical data
models~\cite{Cartmell1985}.

Rabbit's query model lets us rigorously define the basic notions of data
analysis.  Indeed, it can naturally express optional and plural relationships,
database navigation, transitive closure of hierarchical relationships,
aggregate, grouping and data cube operations, query parameters, and window
functions.  All operations on data are represented as query combinators, which
has many practical advantages.  For data analysts, it allows them to build
queries incrementally, validating the output at each step.  For programmers, it
dramatically simplifies GUI-driven query construction.  Finally, Rabbit may be
adapted to a variety of application domains by extending the sets of
primitives, combinators, and (co)monadic containers.

