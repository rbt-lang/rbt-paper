
\section{Query Combinators}
\label{sec:combinators}

In this section, we show how the query model defined in
Section~\ref{sec:cardinality} can support a wide range of operations on data.

\subsection*{Extracting Data}

By traversing the tree of Figure~\ref{fig:unfolded-form}, we can extract data
from the database.

\begin{demo}
    \label{ex:department-name}
    Show the name of each department.
    \begin{equation*}
        \Department\To\Name
    \end{equation*}
\end{demo}

This example is constructed by descending through nodes $\Department$ and
$\Name$, which represent primitives
\begin{alignat*}{3}
    & \Department && : \Void && \to \Seq{\Dept}, \\
    & \Name && : \Dept && \to \Text.
\end{alignat*}
The composition of the primitives inherits the input of the first component and
the output of the second component.  Since one of the components is plural, the
composition is also plural, which gives it a signature
\begin{equation*}
    \Department\To\Name : \Void \to \Seq{\Text}.
\end{equation*}

\begin{demo}
    \label{ex:department-employee-name}
    For each department, show the name of each employee.
    \begin{equation*}
        \Department\To\Employee\To\Name
    \end{equation*}
\end{demo}

This example takes a path through
\begin{alignat*}{3}
    & \Department && : \Void && \to \Seq{\Dept}, \\
    & \Employee && : \Dept && \to \Seq{\Emp}, \\
    & \Name && : \Emp && \to \Text,
\end{alignat*}
to construct a query
\begin{equation*}
    \Department\To\Employee\To\Name : \Void \to \Seq{\Text}.
\end{equation*}

This query produces a list of employee names.  Since each employee belongs to
exactly one department, the list should contain the name of every employee.
The order in which the names appear in the output depends on the intrinsic
order of the $\Department$ and $\Employee$ primitives, but, in any case,
employees within the same department will be coupled together.

The same collection of names, although not necessarily in the same order, is
produced by the following example.

\begin{demo}
    \label{ex:employee-name}
    Show the name of each employee.
    \begin{equation*}
        \Employee\To\Name
    \end{equation*}
\end{demo}

On the other hand, the next example is very different from an apparently
similar Example~\ref{ex:department-name}.

\begin{demo}
    \label{ex:employee-department-name}
    For each employee, show the name of their department.
    \begin{equation*}
        \Employee\To\Department\To\Name
    \end{equation*}
\end{demo}

Here, we should see a list of department names, but each name will appear as
many times as there are employees in the corresponding department.

\begin{demo}
    \label{ex:employee-position}
    Show the position of each employee.
    \begin{equation*}
        \Employee\To\Position
    \end{equation*}
\end{demo}

Similarly, $\Employee\To\Position$ will output duplicate position titles.  We
will see how to produce a list of \emph{unique} positions in
Section~\ref{sec:quotients}.

\begin{demo}
    \label{ex:employee}
    Show all employees.
    \begin{equation*}
        \Employee
    \end{equation*}
\end{demo}

In this example, the query should emit a sequence of entity objects.  In our
Rabbit implementation, entity values are not observable directly, so this
example would instead display a list of records with employee attributes.

\subsection*{Summarizing Data}

Let us see how the extracted data can be summarized.

\begin{demo}
    \label{ex:count-department}
    Show the number of all departments.
    \begin{equation*}
        \Count(\Department)
    \end{equation*}
\end{demo}

This query produces a single number, so that its signature is
\begin{equation*}
    \Count(\Department) : \Void \to \Int.
\end{equation*}
It is constructed by applying the $\Count$ combinator to a query that generates
\emph{a list of all departments}
\begin{equation*}
    \Department : \Void \to \Seq{\Dept}.
\end{equation*}
Comparing the signatures of these two queries, we can derive the signature of
the $\Count$ combinator, in this specific case,
\begin{equation*}
    (\Void \to \Seq{\Dept}) \to (\Void \to \Int),
\end{equation*}
and, in general,
\begin{equation*}
    \Count: (A \to \Seq{B}) \to (A \to \Int).
\end{equation*}
In other words, the $\Count$ combinator transforms any sequence-valued query
to an integer-valued query.  It is implemented by lifting the function that
computes the length of a sequence
\begin{equation*}
    |-| : \Seq{A} \to \Int
\end{equation*}
to a query combinator
\begin{equation*}
    \Count(q) = a \mapsto |q(a)|.
\end{equation*}

We call $\Count$ and other unary combinators that transform a plural query to a
singular (or optional) query \emph{aggregate} combinators.  Some of aggregate
combinators are listed in Table~\ref{tab:common-combinators}.

The next example follows the same pattern.

\begin{demo}
    \label{ex:max-employee-salary}
    Show the highest salary of all employees.
    \begin{equation*}
        \Max(\Employee\To\Salary)
    \end{equation*}
\end{demo}

It extracts the relevant data with
\begin{equation*}
    \Employee\To\Salary : \Void \to \Seq{\Int}
\end{equation*}
and summarizes it using the $\Max$ aggregate
\begin{equation*}
    \Max(\Employee\To\Salary) : \Void \to \Opt{\Int}.
\end{equation*}
This query is optional since it produces no output when the database contains
no employees.

\begin{demo}
    \label{ex:department-count-employee}
    Show the number of employees in each department.
    \begin{equation*}
        \Department\To\Count(\Employee)
    \end{equation*}
\end{demo}

In this example, we convert a plural relationship, \emph{all employees in the
given department},
\begin{equation*}
    \Employee : \Dept \to \Seq{\Emp}
\end{equation*}
to a calculated attribute, \emph{the number of employees in the given
department},
\begin{equation*}
    \Count(\Employee) : \Dept \to \Int.
\end{equation*}
Then we attach it to
\begin{equation*}
    \Department : \Void \to \Seq{\Dept}
\end{equation*}
to get \emph{the number of employees in every department}
\begin{equation*}
    \Department\To\Count(\Employee) : \Void \to \Seq{\Int}.
\end{equation*}

Using the combinator $\Max$, we can further collapse this list to a single
number, as shown in the next example.

\begin{demo}
    \label{ex:max-department-count-employee}
    Show the size of the largest department.
    \begin{equation*}
        \Max(\Department\To\Count(\Employee))
    \end{equation*}
\end{demo}

\subsection*{Pipeline Notation}

Queries are often constructed incrementally, by extracting relevant data and
then shaping it into the desired form with a chain of combinators.  This
construction is made apparent with the \emph{pipeline notation}.

In pipeline notation, the first argument of a combinator is placed in front of
it, separated by colon ($\,\Apply\,$):
\begin{equation*}
    p \Apply F \equiv F(p), \qquad
    p \Apply F(q_1,\ldots,q_n) \equiv F(p,q_1,\ldots,q_n).
\end{equation*}
For example, $\Count(\Department)$ could also be written
\begin{equation*}
    \Department\Apply\Count.
\end{equation*}

A more sophisticated query written in pipeline notation is shown in the
following example.

\begin{demo}
    \label{ex:top-ten-highest-paid-policemen}
    Show the top 10 highest paid employees in the Police department.
    \begin{align*}
        & \Employee \\
        & \Apply\Filter(\Department\To\Name = \textliteral{POLICE}) \\
        & \Apply\Sort(\Salary\Apply\Desc) \\
        & \Apply\Select(\Name,\; \Position,\; \Salary) \\
        & \Apply\Take(10)
    \end{align*}
\end{demo}

Without pipeline notation, this query is much less intelligible:
\begin{multline*}
    \Take(\Select(\Sort(\Filter( \\
    \Employee,\; \Department\To\Name = \textliteral{POLICE}), \\
    \Desc(\Salary)),\; \Name,\; \Position,\; \Salary),\; 10).
\end{multline*}

Combinators $\Filter$, $\Sort$, $\Select$ and $\Take$ used in this query are
described below.

\subsection*{Filtering Data}

We can now demonstrate how to produce entities that satisfy a certain
condition.

\begin{demo}
    \label{ex:filter-by-salary}
    Which employees have a salary higher than \$150k?
    \begin{equation*}
        \Employee\Apply\Filter(\Salary>150000)
    \end{equation*}
\end{demo}

This query introduces several concepts.

First, the integer literal $150000$ represents a primitive query that
\emph{for any given employee, produces the number $150000$}
\begin{equation*}
    150000 : \Emp \to \Int = e \mapsto 150000.
\end{equation*}

Second, the relational symbol ${>}$ denotes a binary combinator, which builds a
query \emph{for a given employee, show whether their salary is higher than
\$150000}
\begin{equation*}
    \Salary > 150000 : \Emp \to \Bool.
\end{equation*}
The combinator
\begin{equation*}
    \placeholder>\placeholder : (A \to \Int,\; A \to \Int) \to (A \to \Bool)
\end{equation*}
is implemented by lifting the relational operator
\begin{equation*}
    \placeholder>\placeholder : (\Int,\; \Int) \to \Bool
\end{equation*}
to an operation on queries
\begin{equation*}
    (p > q) = a \mapsto (p(a) > q(a)).
\end{equation*}
For more examples of constants and regular functions lifted to query primitives
and combinators, see Table~\ref{tab:common-combinators}.

Third, a binary combinator $\Filter$ emits those $\Employee$ entities that
satisfy the predicate query $\Salary > 150000$.  In general, given
\begin{equation*}
    p : A \to \Seq{B}, \qquad q : B \to \Bool
\end{equation*}
a query
\begin{equation*}
    \Filter(p,\; q) : A \to \Seq{B}
\end{equation*}
produces the values of $p$ that satisfy condition $q$
\begin{equation*}
    \Filter(p,\; q) = a \mapsto [\,b \mid b \gets p(a),\; q(b)=\True\,].
\end{equation*}

The following example shows how smoothly $\Filter$ combines with other
combinators.

\begin{demo}
    \label{ex:filter-by-size-and-count}
    How many departments have more than 1000 employees?
    \begin{align*}
        & \Department \\
        & \Apply\Filter(\Count(\Employee)>1000) \\
        & \Apply\Count
    \end{align*}
\end{demo}

\subsection*{Sorting and Paginating Data}

The combinator $\Sort$, applied to a plural query, sorts the query output in
ascending order.

\begin{demo}
    \label{ex:sort-department-name}
    Show the names of all departments in alphabetical order.
    \begin{equation*}
        \Sort(\Department\To\Name)
    \end{equation*}
\end{demo}

The combinator $\Sort$ is implemented by lifting a sequence function
\begin{equation*}
    \Sort : \Seq{A} \to \Seq{A}
\end{equation*}
to a query combinator
\begin{align*}
    & \Sort : (A \to \Seq{B}) \to (A \to \Seq{B}), \\
    & \Sort(p) = a \mapsto \Sort(p(a)).
\end{align*}

\begin{demo}
    \label{ex:sort-employee-by-salary}
    Show all employees ordered by salary.
    \begin{equation*}
        \Employee\Apply\Sort(\Salary)
    \end{equation*}
\end{demo}

In this example, a list of employees is sorted by the value of the attribute
$\Salary$, which is supplied as a second argument to the $\Sort$ combinator.
In this form, $\Sort$ has a signature
\begin{equation*}
    \Sort : (A \to \Seq{B},\; B \to C) \to (A \to \Seq{B}).
\end{equation*}

\begin{demo}
    \label{ex:sort-employee-by-salary-desc}
    Show all employees ordered by salary, highest paid first.
    \begin{equation*}
        \Employee\Apply\Sort(\Salary\Apply\Desc)
    \end{equation*}
\end{demo}

Here, the sort key is wrapped with the combinator $\Desc$ to indicate the
descending sort order.

It is not immediately obvious how to implement $\Desc$ without violating the
query model.  Na\"{\i}vely, $\Desc$ acts like a negation operator, however, not
every type supports negation.  Instead, we can attach a sort order to the type,
so that
\begin{equation*}
    \Int_{\le} \quad\text{and}\quad \Int_{\ge}
\end{equation*}
could indicate the integer type with ascending and descending sort order
respectively.  Then, $\Desc$ could operate by switching the sort order of the
output type and be given a signature
\begin{equation*}
    \Desc : (A \to B) \to (A \to B_{\ge}).
\end{equation*}

\begin{demo}
    \label{ex:sort-employee-by-salary-take-top}
    Who are the top 1\% of highest paid employees?
    \begin{align*}
        & \Employee \\
        & \Apply\Sort(\Salary\Apply\Desc) \\
        & \Apply\Take(\Count(\Employee)\mathbin \div 100)
    \end{align*}
\end{demo}

In this example, only the first 1\% of employees are retained by the combinator
$\Take$, which has two arguments: a query that produces a sequence of employees
\begin{equation*}
    \Employee\Apply\Sort(\Salary\Apply\Desc) : \Void \to \Seq{\Emp}
\end{equation*}
and a query that returns how many employees to take
\begin{equation*}
    \Count(\Employee) \div 100 : \Void \to \Int.
\end{equation*}
Notice that both arguments of $\Take$ have the same input ($\Void$ in this
case), which is reflected in the signature
\begin{equation*}
    \Take : (A \to \Seq{B},\; A \to \Int) \to (A \to \Seq{B}).
\end{equation*}

\subsection*{Query Output}

The combinator $\Select$ customizes the query output.

Previously, we constructed a query to \emph{show the number of employees in
each department} (see Example~\ref{ex:department-count-employee}):
\begin{equation*}
    \Department\To\Count(\Employee).
\end{equation*}
However, this query only produces a list of bare numbers---it does not connect
them to their respective departments.  This is corrected in the following example.

\begin{demo}
    \label{ex:department-select-name-size}
    For every department, show its name and the number of employees.
    \begin{equation*}
        \Department
        \Apply\Select(\Name,\;\Size\As\Count(\Employee))
    \end{equation*}
\end{demo}

In this example, the combinator $\Select$ takes three arguments: the base query
\begin{equation*}
    \Department : \Void \to \Seq{\Dept}
\end{equation*}
and two field queries
\begin{alignat*}{3}
    & \Name && : \Dept && \to \Text, \\
    & \Count(\Employee) && : \Dept && \to \Int.
\end{alignat*}
The $\Select$ combinator generates a sequence of records by applying each field
query to every entity produced by the base query, giving this example a
signature
\begin{equation*}
    \Void \to \Seq{\Tuple{\Name:\Text,\; \Size:\Int}}.
\end{equation*}
The declaration
\begin{equation*}
    \Tuple{\Name:\Text,\; \Size:\Int}
\end{equation*}
defines a \emph{record} type with two fields: a text field $\Name$ and an
integer field $\Size$.  The names of the record fields are derived from the
tags of the field queries, which could be set using the \emph{tagging
notation}.  For example,
\begin{equation*}
    \Size \As \Count(\Employee)
\end{equation*}
binds a tag $\Size$ to the query $\Count(\Employee)$.  Since the tag does not
materially affect the query it annotates, we do not expose the tag in the query
model.

A more complex output structure could be defined with nested $\Select$ combinators.

\begin{demo}
    \label{ex:department-select-name-etc}
    For every department, show the top salary and a list of managers with their
    salaries.
    \begin{alignat*}{4}
        & \Department\hidewidth && && && \\
        & \Apply\Select( && \Name, && && \\
            & && \TopSalary && \As\; && \Max(\Employee\To\Salary), \\
            & && \Manager && \As\; && \Employee \\
            & && && && \Apply\Filter(\Exists(\Subordinate)) \\
            & && && && \Apply\Select(\Name,\; \Salary))
    \end{alignat*}
\end{demo}

In this example, the query output has the type
\begin{align*}
    \keyword{Seq}\{\Tuple{& \Name : \Text,\; \TopSalary : \Opt{\Int}, \\
    & \Manager : \Seq{\Tuple{\Name : \Text,\; \Salary : \Int}}}\}.
\end{align*}

Recall that we could represent the data source in a hierarchical form (see
Figure~\ref{fig:unfolded-form}).  Furthermore, the query output could also be
represented as a hierarchical database, whose structure is determined by the
query signature (see Figure~\ref{fig:department-select-name-etc}).  Thus,
a query could be seen as transforming one hierarchical database to another.


\begin{figure}
    \centering
    \begin{forest}
        unfolded database
        [,void
            [$\Department$,map,plural
                [$\Name$,map,singular]
                [$\TopSalary$,map,optional]
                [$\Manager$,map,plural
                    [$\Name$,map,singular],
                    [$\Salary$,map,singular]]]]
    \end{forest}
    \caption{Output database for Example~\ref{ex:department-select-name-etc}}
    \label{fig:department-select-name-etc}
\end{figure}



\subsection*{Query Aliases}

A complex query could often be simplified by replacing duplicate expressions
with aliases.

\begin{demo}
    \label{ex:department-define-size}
    Show the top 3 largest departments and their sizes.
    \begin{align*}
        & \Department \\
        & \Apply\Define(\Size\As\Count(\Employee)) \\
        & \Apply\Sort(\Size\Apply\Desc) \\
        & \Apply\Select(\Name,\; \Size) \\
        & \Apply\Take(3)
    \end{align*}
\end{demo}

In this example, the alias $\Size$ is created in two steps: first, the query
\begin{equation*}
    \Count(\Employee): \Dept \to \Int
\end{equation*}
is tagged with the name $\Size$, and then $\Size$ is added to scope of $\Dept$
by the combinator $\Define$.

Although this query could have been written as
\begin{align*}
    & \Department \\
    & \Apply\Sort(\Count(\Employee)\Apply\Desc) \\
    & \Apply\Select(\Name,\; \Count(\Employee)) \\
    & \Apply\Take(3),
\end{align*}
the use of an alias makes this example more legible, not only by reducing
redundancy, but also by assigning a name to a key concept of the query.

\subsection*{Hierarchical Relationships}

Hierarchical relationships are encoded by primitives with identical types of
input and output.

For example, the relationship between an employee and their manager is
expressed with
\begin{equation*}
    \Manager : \Emp \to \Opt{\Emp}.
\end{equation*}

\begin{demo}
    \label{ex:employee-filter-salary-manager}
    Find all employees whose salary is higher than the salary of their manager.
    \begin{equation*}
        \Employee\Apply\Filter(\Salary>\Manager\To\Salary)
    \end{equation*}
\end{demo}

This example uses familiar combinators $\Filter$ and ${>}$ (see
Example~\ref{ex:filter-by-salary}), but an alert reader will notice the
disagreement between the signature of the combinator
\begin{equation*}
    \placeholder>\placeholder : (A \to \Int,\; A \to \Int) \to (A \to \Bool)
\end{equation*}
and the signatures of its arguments
\begin{alignat*}{3}
    & \Salary && : \Emp && \to \Int, \\
    & \Manager\To\Salary && : \Emp && \to \Opt{\Int}.
\end{alignat*}
Namely, ${>}$ expects its arguments to be singular, but the output of
$\Manager\To\Salary$ is optional.

To legitimize this query, we adopt the following rule.  When one argument of a
scalar combinator has a non-trivial cardinality, this cardinality can be
promoted to the output of the combinator.  This rule gives ${>}$ a signature
\begin{equation*}
    \placeholder>\placeholder : (A \to \Int,\; A \to M\{\Int\}) \to (A \to M\{\Bool\})
\end{equation*}
or, in this specific case,
\begin{equation*}
    \Salary > \Manager\To\Salary : \Emp \to \Opt{\Bool}.
\end{equation*}
Finally, we need to let $\Filter$ accept predicate queries with optional
output, by treating $\bot$ as $\False$.

Using expressions
\begin{align*}
    & \Manager, \\
    & \Manager\To\Manager, \\
    & \Manager\To\Manager\To\Manager,\; \ldots
\end{align*}
we can build queries that involve the manager, the manager's manager, etc.  We
can also obtain \emph{the complete chain of command for the given employee}
with
\begin{equation*}
    \Connect(\Manager) : \Emp \to \Seq{\Emp}.
\end{equation*}

\begin{demo}
    \label{ex:employee-salary-vs-superior}
    How does employees' salary compare to the average of their superiors?
    \begin{alignat*}{2}
        & \Employee\hidewidth && \\
        & \Apply\Select(&& \Name, \\
        & && \Salary-\Mean(\Connect(\Manager)\To\Salary))
    \end{alignat*}
\end{demo}

The query
\begin{equation*}
    \Connect(\Manager)\To\Salary : \Emp \to \Seq{\Int}
\end{equation*}
produces \emph{the salaries of all managers above the given employee}.  They
are collapsed to a single number with the combinator $\Mean$.

In general, the combinator $\Connect$ maps an optional self-referential query to a
plural self-referential query by taking its transitive closure:
\begin{multline*}
    \Connect : (A \to \Opt{A}) \to (A \to \Seq{A}), \\
    \shoveleft{\Connect(p) = a \mapsto [\;p(a),\; p(p(a)),\; \ldots,\; p^{(n)}(a)\;]} \\
    (p^{(n)}(a) \ne \bot,\; p^{(n+1)}(a) = \bot).
\end{multline*}


\begin{table}
    \begin{framed}
    \textbf{Identity and constants}
    \begin{alignat*}{3}
        & \Here && : A \to A && = a \mapsto a \\
        & 150000 && : A \to \Int && = a \mapsto 150000 \\
        & \Null && : A \to \Opt{B} && = a \mapsto \bot
    \end{alignat*}
    \textbf{Some scalar combinators}
    \begin{alignat*}{3}
        & {=},\;{\ne} && : (A \to B,\; A \to B) && \to (A \to \Bool) \\
        & {<},\;{\le},\;{>},\;{\ge} && : (A \to B,\; A \to B) && \to (A \to \Bool) \\
        & {\&},\;{|} && : (A \to \Bool,\; A \to \Bool) && \to (A \to \Bool) \\
        & {+},\;{-} && : (A \to \Int,\; A \to \Int) && \to (A \to \Int) \\
        & \Length && : (A \to \Text) && \to (A \to \Int)
    \end{alignat*}
    \textbf{Aggregate combinators}
    \begin{alignat*}{3}
        & \Count && : (A \to \Seq{B}) && \to (A \to \Int) \\
        & \Exists && : (A \to \Seq{B}) && \to (A \to \Bool) \\
        & \Any,\; \All && : (A \to \Seq{\Bool}) && \to (A \to \Bool) \\
        & \Sum && : (A \to \Seq{\Int}) && \to (A \to \Int) \\
        & \Max,\; \Min && : (A \to \Seq{\Int}) && \to (A \to \Opt{\Int})
    \end{alignat*}
    \textbf{Sequence transformers}
    \begin{alignat*}{4}
        & \Filter && : (&& A \to \Seq{B},\; B \to \Bool) && \to (A \to \Seq{B}) \\
        & \Sort && : (&& A \to \Seq{B},\; && \\
        & && && B \to C_1,\; \ldots,\; B \to C_n) && \to (A \to \Seq{B}) \\
        & \Take && : (&& A \to \Seq{B},\; A \to \Int) && \to (A \to \Seq{B}) \\
        & \Unique && : (&& A \to \Seq{B}) && \to (A \to \Seq{B})
    \end{alignat*}
    \textbf{Selector and modifiers}
    \begin{alignat*}{2}
        & \Select & : (& A \to M\{B\}, \\
        & && B \to M_1\{C_1\},\; \ldots,\; B \to M_n\{C_n\}) \\
        & \;\to (A \to M\{\Tuple{M_1\{C_1\},\; \ldots,\; M_n\{C_n\}}\})\hidewidth && \\
        & \Define & : (& A \to M\{B\},\; B \to T) \to (A \to M\{B\}) \\
        & \Asc,\; \Desc\; & : (& A \to B) \to (A \to B_\lessgtr)
    \end{alignat*}
    \textbf{Hierarchical connector}
    \begin{equation*}
        \Connect : (A \to \Opt{A}) \to (A \to \Seq{A})
    \end{equation*}
    \textbf{Grouping}
    \begin{alignat*}{2}
        & \Group && : ( A \to \Seq{B},\; B \to C_1,\; \ldots,\; B \to C_n) \\
        & \;\to (A && \to \Seq{\Tuple{C_1,\; \ldots,\; C_n,\; \Seq{B}}}) \\
        & \Rollup && : ( A \to \Seq{B},\; B \to C_1,\; \ldots,\; B \to C_n) \\
        & \;\to (A && \to \Seq{\Tuple{\Opt{C_1},\; \ldots,\; \Opt{C_n},\; \Seq{B}}})
    \end{alignat*}
    \textbf{Context primitives and combinators}
    \begin{alignat*}{2}
        & \Before,\; \Around : \Rel{A} \to \Seq{A} \\
        & \Given : (\Ctx{T}{A} \to M\{B\},\; A \to T) \to (A \to M\{B\}) \\
        & \textit{\textsf{PARAM}} : \Ctx{T}{A} \to T
    \end{alignat*}
    \end{framed}
    \caption{Some primitives and combinators}
    \label{tab:common-combinators}
\end{table}



